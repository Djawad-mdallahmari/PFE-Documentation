\documentclass[UTF8]{EPURapport}
%\usepackage{listings}

%\renewcommand{\lstlistlistingname}{Liste des codes}
%\renewcommand{\lstlistingname}{Code}

%\addextratables{%
%	\lstlistoflistings
%}

%\swapAuthorsAndSupervisors

\thedocument{Rapport de projet}{Canne connectée pour aveugles}{}
\grade{Département Informatique\\ 5\ieme{} année\\ 2020-2021}
\authors{%
	\category{Auteurs}{%
		\name{Djawad M'DALLAH MARI} \mail{djawad.mdallah-mari@etu.univ-tours.fr}
	}
	\details{DII5 2020-2021}
}
\supervisors{%
	\category{Encadrants}{%
		\name{Gilles VENTURINI} \mail{gilles.venturini@etu.univ-tours.fr}
	}
	\details{Université François-Rabelais, Tours}
}
\abstracts{Rapport du projet canne connecée pour aveugles}
{}
{}
{}

\begin{document}

\chapter{Remerciements}

Premièrement, je voudrais adresser mes remerciements à mon encadrant de projet, \textbf{M. Gilles VENTURINI}, initiateur de ce projet. Ses conseils et recommandations m'ont permis de mener au mieux ce projet. En outre, je voudrais remercier l'ensemble de l'équipe pédagogique de \textbf{Polytech Tours}, pour leurs efforts et engagements durant ces périodes scolaires. Leurs efforts m'ont apporté connaissances et compétences indispensables que j'ai pu mettre en pratique durant la réalisation de ce projet. Je tiens également à remercier mes camarades pour leur soutien et leur collaboration durant ces années d'études. 

\chapter{Introduction}
Ce document est le rapport du projet intitulé "canne connectée pour aveugles" réalisé dans le cadre d'un projet de fin d'étude (PFE) à Polytech Tours. Elle vise à synthétiser le travail réalisé en montrant les différents aspects du projet tels que la gestion organisationnelle, technique et humaine.\\

Ce projet de fin d'étude est donc un projet qui vise à mettre en pratique les acquis de ces dernières années, en mettant en particulier un accent sur les capacités d'analyses et de réflexion ainsi que la rigueur des solutions proposées pour répondre aux problématiques.\\

Nous verrons donc dans ce document les problèmes posés, les enjeux associés et les réflexions menées pour aboutir à des solutions. Nous verrons dans un premier temps une présentation du projet ainsi que ces objectifs. Nous aborderons ensuite la stratégie d'approche servant d'axe principal pour mener ce projet. Après, nous verrons les méthodes et outils de gestion de projet qui ont été mis en place. Les choix et réalisations seront ensuite présentés suivi d'une prise de recul détaillant notamment les points critiques et les difficultés rencontrées lors du projet. Enfin, une conclusion sera faite apportant quelques remarques personnelles sur le projet.

\chapter{Présentation}

\section{Contexte}

\section{Objectifs}

\chapter{Stratégie d'approche}

\section{Communication}

\section{Fournir des solutions adaptées}

\section{Faciliter la reprise du projet}

\chapter{Gestion du projet}

\section{Plannings}

\section{Gestion des tâches}

\section{Point d'avancement}

\section{Versionning}

\chapter{Choix et réalisations}

\section{Intégration du modèle}

\section{Synthétiseur vocal}

\section{Viseur virtuel}

\section{Start and Stop}

\section{Capteurs}

\section{Documentations}

\chapter{Prise de recul}

\section{Points positifs}

\section{Points critiques}

\section{Difficultés}

\section{Améliorations possibles}

\chapter{Conclusion}

\annexes

\end{document}