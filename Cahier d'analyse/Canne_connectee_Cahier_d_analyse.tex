\documentclass[UTF8]{EPURapport}
\input{include.tex}
\thedocument{Cahier d'analyse}{Canne connectée pour aveugles}{}
\grade{Département Informatique\\ 5\ieme{} année\\ 2020-2021}
\authors{%
	\category{Auteurs}{%
		\name{Djawad M'DALLAH MARI} \mail{djawad.mdallah-mari@etu.univ-tours.fr}
	}
	\details{DII5 2020-2021}
}
\supervisors{%
	\category{Encadrants}{%
		\name{Gilles VENTURINI} \mail{gilles.venturini@etu.univ-tours.fr}
	}
	\details{Université François-Rabelais, Tours}
}
\abstracts{Cahier d'analyse canne connecée pour aveugles}
{}
{}
{}

\begin{document}

\chapter{Cahier d'analyse}

\section{Introduction}
Ce cahier d'analyse s'inscrit dans le cadre du projet Canne connectée pour aveugles. Il vise à présenter les analyses faites pour répondre aux besoins exprimés dans le cahier de spécifications. Une lecture au préalable du cahier de spécifications est donc recommandée afin de comprendre le contexte et les enjeux du projet.

Nous verrons donc dans ce document une analyse sur l'application Android à développer. Nous verrons en particulier quelques méthodes de reconnaissances d'objet pour le mobile, la méthode qui sera mise en place pour faire de la synthèse vocale et également comment garantir à l'utilisateur une interface adaptée à ses contraintes.

\section{Reconnaissance d'objet}
\subsection{TensorFlow}
\subsection{Modèles}
\subsection{Banque d'images}

\section{Synthèse vocale}
\subsection{subsection}

\section{Navigation}
\subsection{subsection}

\annexes

\end{document}